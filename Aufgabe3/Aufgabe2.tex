\documentclass{article}
\usepackage{german}
\author{Dominique Ostermayer} 
\title{CS102 \LaTeX	\"Ubung} 
\date{\today}
\begin{document}
\maketitle

\section{Das ist der erste Abschnitt}
Dieser Abschnitt ist leer. (Hier steht, dass hier nichts steht.)

\section{Tabelle}
Die untenstehende Tabelle 1 zeigt empirisch ermittelte Daten zu Ikonen und Nichtikonen.

\begin{table}[h]
\centering
\begin{tabular}{c|c|c|c}
\hspace{3 pt} & Popkulturelle Frequenz & Ikonenstatus & Erfolg in \% \\
\hline
Warhol & $\infty$ & 10 & 98 \\
Louis XIV & 2835 & 8 & 39 \\
Kanye West & ? & 0 & N/A \\
Clinton & 9873 & 6 & 73
\end{tabular}
\caption{Popkulturelle W\"urdentr\"ager im Vergleich}
\end{table}

\section{Formeln}
\subsection{Pythagoras}
Der Satz des Pythagoras errechnet sich wie folgt: $a^2 + b^2 = c^2$. Durch Umformen erh\"alt man die L\"ange der Hypothenuse: $c = \sqrt[]{a^2 + b^2}$ \\
\subsection{Summen}
Wir k\"onnen auch die Formel f\"ur eine Summe angeben: \\

$s = \sum\limits_{i=1}^n i = \frac{n \cdot (n + 1)}{2}$
\centering
   
\end{document} 